\chapter{Method}
\section{Research Design}

\section{Content Analysis}
Having obtained seventy-seven retrospectives spanning over five years, the means of analyzing them had to be determined. 

The retrospectives consisted of five different sections being: Where and What, Actions, Comments, Signatures and Case Proceedings, as can be seen in \autoref{table:retrospective-format}. 
The ``Where and What'' section contained general data about the retrospective such as the date, iteration dates, iteration number, contact person and other general information. ``Actions'' described the improvement actions that had resulted from the retrospective. It contained a description for each action, who is responsible for that action, deadline, status and comments from the participants. The ``Comments'' section contained comments, if any, from the participants of the retrospective if they had any specific input for the retrospective in general. ``Signatures'' contained the signature for each participant in the retrospective. The last section,``Case Proceedings'' contained information about the changes in the document and circulation of it.

\begin{table}[!h]
	\begin{center}
	\caption{The section of the retrospectives}
	\label{table:retrospective-format}
	\makebox[\textwidth]{%
		\begin{tabular}{l | p{0.7\textwidth}}
		\hline
		Part & Description \\
		\hline
		Where and What & Containing general data about the retrospective such as date, iteration location, etc. \\
		Actions & Describes the actions resulting from the retrospective it also includes data on responsible person, deadline etc. \\
		Comments & General comments from the participants for the retrospectives. \\
		Signatures & The signatures from each participant participating in the retrospective. \\
		Case Proceedings & The changelog and circulation of the retrospective. \\
		\hline
		\end{tabular}
	}
\end{center}
\end{table}

While getting familiar with the retrospectives we found that the only section containing any value in the terms of organizational learning was the actions described in the ``Action'' section. In most of the retrospective multiple actions where described relating to different issues observed during the iteration. The format of the actions can be seen in \autoref{table:action-format}. 

\begin{table}[!h]
	\begin{center}
	\caption{An generic example of an action provided in the retrospectives.}
	\label{table:action-format}
	\makebox[\textwidth]{%
		\begin{tabular}{| l  p{0.5\textwidth} |}
		\hline
		Action x &  \\
		Deadline & 01.01.2015 \\
		Action description & Always add a week to the iterations during holidays. \\
		Comments & Resources are not reliable during holidays. \\
		Responsible unit & Team X \\
		User responsible for action & John Smith \\
		Status & Completed or In Process \\
		Completed & 31.01.2014 \\
		Type & Preventive or Corrective \\
		\hline
		\end{tabular}
	}
\end{center}
\end{table}

To retrieve any research-worthy knowledge from the many actions given by the retrospectives we wanted to make the actions comparable. We settled on tabulations as our analysis method of the actions. 

Tabulations provide easy means of rendering data comprehensible. Krippendorff ~\cite{Krippendorff2004} describes tabulations as: 

\begin{quote}
Tabulation refers to collecting same or similar recording units in categories and presenting counts of how many instances are found in each. Tabulations produce tables of absolute frequencies, such as the number of words in each category occurring in a body of text, or of relative frequencies, such as percentages expressed relative to the sample size, proportions of a total, or probabilities.
\end{quote}

For our case we are going use absolute frequencies to count the occurrences of different categories described in \autoref{method:categories}. Using relative frequencies would not suffice in our case where determining whether an action is twenty percent technical and eighty percent procedural or thirty percent technical and seventy percent procedural would be immensely difficult and not to mention impractical. Rather an action could be neither technical or procedural,be one of them, or be both. Thus resulting us using absolute frequencies when we are counting occurrences of the different categories.

\subsection{Pilot Analysis}
Settling on tabulations as our means of content analysis the different categories had to determined. We performed a pilot

\subsection{Categories}\label{method:categories}
The results of our pilot analysis gave us an extended set of categories that we will now describe further. Mainly we found six main themes that we could derive categories from. The six themes were: Nature, Context, Decision Making, Organizational Learning, Development and Management. We will describe each of these themes and their set of categories further in the sections below. A complete list is shown in \autoref{table:content-analysis-categories}

\begin{table}[!h]
	\begin{center}
		\caption{The final set of themes and categories that the content analysis is based upon.}
		\label{table:content-analysis-categories}
		\begin{adjustbox}{width=\textwidth, totalheight=0.95\textheight, keepaspectratio}
			\begin{tabular}{| l | l | p{0.5\textwidth} |}
			\hline
			Theme & Category & Short Description  \\
			\hline
			Nature & Positive & If an action is a result of a continuation of a process \\ \cline{2-3}
			& Negative & Action is a result of a arisen problem \\ \cline{2-3}
			& Undefined & The nature of the action is unclear or undefined \\ \hline
			Context & Technical & If the action is related to some technical context. \\ \cline{2-3}
			& Process & Action is related to a process context \\ \cline{2-3}
			& Undefined & The action could not be related to either technical or process \\ 
			\hline
			Decision Making & Strategic & Action is suggesting long-term change \\ \cline{2-3}
			& Tactical & The action is related to identification and use of resources. \\ \cline{2-3}
			& Operational & If the action is ensuring effectiveness and day-to-day operations \\ \cline{2-3}
			& Undefined & If the action is unclear and doesn't fit any of the other decision making categories. \\ 
			\hline 
			Organizational learning & Single-loop & If the action do a change that only influence the effects \\ \cline{2-3}
			& Double-loop & If one understand the factors that influence effects, and the nature of this influence. \\ \cline{2-3}
			& Undefined & If the action unclear in terms of organizational learning \\ \hline
			Development Phase & Development & Action is related to the development phase \\ \cline{2-3}
			& Testing & The action is related to testing \\ \cline{2-3}
			& Documentation & Action is related to Documentation \\ \cline{2-3}
			& Builds & The action is related to building of software systems \\ \cline{2-3}
			& Release & The action is related to releasing of software \\ \cline{2-3}
			& Business & The action is related to business development \\ \cline{2-3}
			& Undefined & The action is not related to any of the development phases described above \\ 
			\hline
			Management & Communication & Related to communication within a team \\ \cline{2-3}
			& Leadership & Action is related to leadership \\ \cline{2-3}
			& Incompetence & Action is a result of lacking knowledge or experience \\ \cline{2-3}
			& External relations & The action is related to customer relations or other external stakeholders \\ \cline{2-3}
			& Planning & The action is a result of bad or good planning \\ \cline{2-3}
			& Undefined & Action is not related to any of the management issues \\
			\hline
			\end{tabular}
		\end{adjustbox}
	\end{center}
\end{table}

\subsubsection{Nature}
\paragraph{Positive}
\paragraph{Negative}
\paragraph{Undefined}
\subsubsection{Context}
\paragraph{Technical}
\paragraph{Process}
\paragraph{Undefined}
\subsubsection{Decision Making}
\paragraph{Strategic}
\paragraph{Tactical}
\paragraph{Operational}
\paragraph{Undefined}
\subsubsection{Organizational Learning}
\paragraph{Single-loop}
\paragraph{Double-loop}
\paragraph{Undefined}
\subsubsection{Development}
\paragraph{Development}
\paragraph{Testing}
\paragraph{Documentation}
\paragraph{Builds}
\paragraph{Release}
\paragraph{Business}
\paragraph{Undefined}
\subsubsection{Management}
\paragraph{Communication}
\paragraph{Leadership}
\paragraph{Incompetence}
\paragraph{External Relations}
\paragraph{Planning}
\paragraph{Undefined}
\subsection{Framework Limitations}
\subsubsection{Missing Contexts}
\subsubsection{Borderline Actions}
\subsection{Processing Steps}

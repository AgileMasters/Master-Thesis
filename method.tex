\chapter{Method}
\section{Research Design}

\section{Content Analysis}
Having obtained seventy-seven retrospectives spanning over five years, the means of analyzing them had to be determined. 

The retrospectives consisted of five different sections being: Where and What, Actions, Comments, Signatures and Case Proceedings, as can be seen in \autoref{table:retrospective-format}. 
The ``Where and What'' section contained general data about the retrospective such as the date, iteration dates, iteration number, contact person and other general information. ``Actions'' described the improvement actions that had resulted from the retrospective. It contained a description for each action, who is responsible for that action, deadline, status and comments from the participants. The ``Comments'' section contained comments, if any, from the participants of the retrospective if they had any specific input for the retrospective in general. ``Signatures'' contained the signature for each participant in the retrospective. The last section,``Case Proceedings'' contained information about the changes in the document and circulation of it.

\begin{table}[!h]
	\begin{center}
	\caption{The section of the retrospectives}
	\label{table:retrospective-format}
	\makebox[\textwidth]{%
		\begin{tabular}{l | p{0.7\textwidth}}
		\hline
		Part & Description \\
		\hline
		Where and What & Containing general data about the retrospective such as date, iteration location, etc. \\
		Actions & Describes the actions resulting from the retrospective it also includes data on responsible person, deadline etc. \\
		Comments & General comments from the participants for the retrospectives. \\
		Signatures & The signatures from each participant participating in the retrospective. \\
		Case Proceedings & The changelog and circulation of the retrospective. \\
		\hline
		\end{tabular}
	}
\end{center}
\end{table}

While getting familiar with the retrospectives we found that the only section containing any value in the terms of organizational learning was the actions described in the ``Action'' section. In most of the retrospective multiple actions where described relating to different issues observed during the iteration. The format of the actions can be seen in \autoref{table:action-format}. 

\begin{table}[!h]
	\begin{center}
	\caption{An generic example of an action provided in the retrospectives.}
	\label{table:action-format}
	\makebox[\textwidth]{%
		\begin{tabular}{| l  p{0.5\textwidth} |}
		\hline
		Action x &  \\
		Deadline & 01.01.2015 \\
		Action description & Always add a week to the iterations during holidays. \\
		Comments & Resources are not reliable during holidays. \\
		Responsible unit & Team X \\
		User responsible for action & John Smith \\
		Status & Completed or In Process \\
		Completed & 31.01.2014 \\
		Type & Preventive or Corrective \\
		\hline
		\end{tabular}
	}
\end{center}
\end{table}

To retrieve any research-worthy knowledge from the many actions given by the retrospectives we wanted to make the actions comparable. We settled on tabulations as our analysis method of the actions. 

Tabulations provide easy means of rendering data comprehensible. Krippendorff ~\cite{Krippendorff2004} describes tabulations as: 

\begin{quote}
Tabulation refers to collecting same or similar recording units in categories and presenting counts of how many instances are found in each. Tabulations produce tables of absolute frequencies, such as the number of words in each category occurring in a body of text, or of relative frequencies, such as percentages expressed relative to the sample size, proportions of a total, or probabilities.
\end{quote}

For our case we are going use absolute frequencies to count the occurrences of different categories described in \autoref{method:categories}. Using relative frequencies would not suffice in our case where determining whether an action is twenty percent technical and eighty percent procedural or thirty percent technical and seventy percent procedural would be immensely difficult and not to mention impractical. Rather an action could be neither technical or procedural,be one of them, or be both. Thus resulting us using absolute frequencies when we are counting occurrences of the different categories.

\subsection{Pilot Analysis}
Settling on tabulations as our means of content analysis the different categories had to determined. We performed a pilot study.
The pilot study was conducted in order to investigate the potential of analyzing a set of the 77 retrospective reports.  The pilot study was limited to 11 reports, where we picked every 7th retrospective chronologically. This distribution was chosen in order to get an even spread to represent the whole set, as well as keep the size manageable for the short preliminary study.

The pilot study analysis lasted for one week, and included agreeing on the parameters and methods for the study, as well as a short workshop session where the results were presented in front of a group of fellow researchers. This workshop session consisted of a short presentation of the findings of the study. After the presentation we had a brainstorming session where we received feedback on potential improvements, as well as general impressions. 

\begin{figure}[!h]
	\centering
	\includegraphics[width=\textwidth]{figures/Pilot_loop.PNG}
	\caption{An example of a slide from the pilot study presentation }
	\label{figure:Pilot Slide}
\end{figure}

\subsection{Categories}\label{method:categories}
The results of our pilot analysis gave us an extended set of categories that we will now describe further. Mainly we found six main themes that we could derive categories from. The six themes were: Nature, Context, Decision Making, Organizational Learning, Development and Management. We will describe each of these themes and their set of categories further in the sections below. A complete list is shown in \autoref{table:content-analysis-categories}. 

\begin{table}[h]
	\begin{center}
		\caption{The final set of themes and categories that the content analysis is based upon.}
		\label{table:content-analysis-categories}
		\begin{adjustbox}{width=\textwidth, totalheight=0.9\textheight, keepaspectratio}
			\begin{tabular}{| l | l | p{0.5\textwidth} |}
			\hline
			Theme & Category & Short Description  \\
			\hline
			Nature & Positive & If an action is a result of a continuation of a process \\ \cline{2-3}
			& Negative & Action is a result of a arisen problem \\ \cline{2-3}
			& Undefined & The nature of the action is unclear or undefined \\ \hline
			Context & Technical & If the action is related to some technical context. \\ \cline{2-3}
			& Process & Action is related to a process context \\ \cline{2-3}
			& Undefined & The action could not be related to either technical or process \\ 
			\hline
			Decision Making & Strategic & Action is suggesting long-term change \\ \cline{2-3}
			& Tactical & The action is related to identification and use of resources. \\ \cline{2-3}
			& Operational & If the action is ensuring effectiveness and day-to-day operations \\ \cline{2-3}
			& Undefined & If the action is unclear and doesn't fit any of the other decision making categories. \\ 
			\hline 
			Organizational learning & Single-loop & If the action do a change that only influence the effects \\ \cline{2-3}
			& Double-loop & If one understand the factors that influence effects, and the nature of this influence. \\ \cline{2-3}
			& Undefined & If the action unclear in terms of organizational learning \\ \hline
			Development Phase & Development & Action is related to the development phase \\ \cline{2-3}
			& Testing & The action is related to testing \\ \cline{2-3}
			& Documentation & Action is related to Documentation \\ \cline{2-3}
			& Builds & The action is related to building of software systems \\ \cline{2-3}
			& Release & The action is related to releasing of software \\ \cline{2-3}
			& Business & The action is related to business development \\ \cline{2-3}
			& Undefined & The action is not related to any of the development phases described above \\ 
			\hline
			Collaboration & Communication & Related to communication within a team \\ \cline{2-3}
			& Leadership & Action is related to leadership \\ \cline{2-3}
			& Competence & Action is a result of lacking knowledge or experience \\ \cline{2-3}
			& External relations & The action is related to customer relations or other external stakeholders \\ \cline{2-3}
			& Planning & The action is a result of bad or good planning \\ \cline{2-3}
			& Undefined & Action is not related to any of the collaboration issues \\
			\hline
			\end{tabular}
		\end{adjustbox}
	\end{center}
\end{table}
\afterpage{\clearpage}

\subsubsection{Nature}
The nature of the action is the first theme that the content analysis is going to inspect. We define the nature of the action as how the origins of the action began. Did they come from a problem that occurred during the iteration or is it a continuation of something that has been working well in the past. Through our analysis we will try to understand the origins of the actions and classify them either as positive, negative or undefined. We define what we classify as what below. 
\paragraph{Positive} Positive actions is those actions where the origins of the action is in a positive context. If the action represents a current good working practice being continued, or something uncommon happened that gave positive results, it would be classified as positive.
\paragraph{Negative} The negative actions are those actions that has its origins from a problem or abnormal issue resulting in negative results. If an action is a result of a problem or abnormal issue it is classified as an negative action. 
\paragraph{Undefined} In the cases where it is unclear whether the origins of the action is positive or negative we classify the action as undefined. Such occurrences can be a result of missing context or actions that seem to have neither positive issues or negative issues as its origin. 
\subsubsection{Context}	
The context surrounding the action will be analyzed. The context off the action is based on the underlying issue that leads to the needed action. We divide the issues into three main categories, technical, process and undefined. 
\paragraph{Technical}
A technical issue can be a issue relating to technical competence, bugs or problems.
\paragraph{Process}
A process issue stems from a problem with a process, or potential for improvements in the existing processes. This can for example relate to communication between colleagues or work scheduling.
\paragraph{Undefined}
 An undefined issue might not have any clear origin, or might be too loosely described to be classified easily.
\subsubsection{Decision Making}
The scope of a decision will be analyzed, is it a short and quick fix or a long term large task. The categories are strategic, tactical, operational and undefined. 
\paragraph{Strategic}
A strategic decision is a wide ranging decision dealing with multiple or sizable issues, often causing major changes and have a long term impact. 
\paragraph{Tactical}
A tactical decision is smaller than a strategic decision that seeks to deal with the distribution and use of resources available to the team. 
\paragraph{Operational}
Operational decisions deal with ensuring effectiveness of day-to-day operations within the organization. They might be quick fixes that solve a single problem.
\paragraph{Undefined}
An undefined decision might be difficult to categorize because of a lack of context or an unclear description. 
\subsubsection{Organizational Learning}
Organizational learning is a process where an organization takes steps to improve its current work environments by reacting to issues that arise. These steps can be varied, and we divide them into single-loop, double-loop and undefined. 
\paragraph{Single-loop}
A single-loop action is an action designed to change or tune a process in order to improve it. The action does not seek to address underlying problems, and are a single-feedback loop from observing an issue to making a change.
\paragraph{Double-loop}
A double-loop action is designed to solve an issue, as well as address the underlying cause of the issue. This requires an understanding of the underlying issue and the nature of its influence. 
\paragraph{Undefined}
An action might not be clearly described, or the nature of the action can not be interpreted. We will classify these actions as undefined.
\subsubsection{Development}
\paragraph{Development}
\paragraph{Testing}
\paragraph{Documentation}
\paragraph{Builds}
\paragraph{Release}
\paragraph{Business}
\paragraph{Undefined}
\subsubsection{Collaboration}
One of the aspects during a software developing process is collaboration. Fægri ~\cite{Faegri2012} describes collaboration as the following: 
\begin{quote}
Collaboration is a key aspect of software development. Collaboration allows groups of software practitioners to deal with uncertainty, complexity and interdependence. And in dealing with these challenges, the group demonstrates its collective problem-solving ability.
\end{quote}
Through our pilot analysis we registered several activities that are related to collaboration. Communication, leadership, competence, external relations and planning all belongs beneath the collaboration banner, and we describe in detail how we classify each of them below. 
\paragraph{Communication}
Communication is a widely used word and concept, but it rarely is defined. By using Merriam-Webster dictionary ~\cite{MerriamWebster2015} we found a definition that can serve as a clarification: 
\begin{quote}
The act or process of using words, sounds, signs, or behaviors to express or exchange information or to express your ideas, thoughts, feelings, etc., to someone else.
\end{quote}
Nakakoji et al. ~\cite{Nakakoji2010} distinguishes between two types of communication related to software development, coordination communication and expertise communication. The first one being the process of coordinating the development activities and the last one being when a developer obtain some information regarding a software artifact, either through code comments, wikis or other means. 
We are however not going to distinguish between these two communication types in our content analysis as we believe the two differentiations is covered by the context of the action as described in \todo[inline]{Cite to Context section}. For our content analysis we are simply going to count every instance of communication for every action that is related to communication between team-members regardless if it is through text, speech or other means of communication.
\paragraph{Leadership}
As is the case with communication, leadership is also widely used concept, that is rarely specified. Again we turn to Merriam-Webster \cite{MerriamWebster2015} for a definition: 
\begin{quote}
The power or ability to lead other people.
\end{quote}
Agile development teams is often self-organized as it is one of the principles in the agile manifesto \cite{AgileManifesto2015}. This can result in no clear leadership. For our content analysis we are going to consider decisions and guidelines set by the group itself as leadership activities. We categorize actions as leadership if they somehow suggests changes to leadership or if the actions is a result of some leadership related issue. 
\paragraph{Competence}
We define competence as the ability to perform a certain task in a adequate quality. Each member within an agile team has their own set of knowledge that they use to solve different tasks. If any issues arises and the group is lacking the knowledge to counter it we categorize the action created to resolve it as an competence action. An example could be issue of lacking the knowledge resolving a database error and the action is to send one developer at a database course. This action would then be categorized as an competence action.
\paragraph{External Relations}
By the category external relations we mean if the team has any actions that is a result of issues arising from external factors or the actions that team is creating to inflict some external factors. Example of external relations can be customer relations, communication with server maintenance team or other development teams.  
\paragraph{Planning}
An action that is categorized as planning is when the action is suggesting changes to planning in future iterations or is a result of an issue occurring in a former iteration. Estimations, task-prioritization, scheduling and etc. all goes under the planning category. 
\paragraph{Undefined}
For those actions that the origin is unclear or the goal is not related to any of the collaboration categories we categorize them as undefined.
\subsection{Content Analysis Limitations}
\subsubsection{Missing Contexts}
\subsubsection{Borderline Actions}
\subsection{Processing Steps}

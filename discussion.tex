\chapter{Discussion}
\section{Retrospectives}
\section{Organizational Learning} % (fold)
\label{sec:organizational_learning}
\subsection{Governing Values}
Argyris and Schön\cite{Argyris1996} described several governing values for learning organizations as we described in \autoref{intro:organizational-learning}. Throughout this section we will reflect on our results using these governing values investigating how retrospectives is performing as a learning practice. 

In Model I, described in \autoref{sub:model_i}, four governing values set the focus for the learning organization. Into our investigation of the retrospective practice we have seen very little to any of these values. 

One could argue that the team setting goals and achieving them could be compared to creating actions and fulfilling them, however we do not support this as the fulfillment of actions is part of a collective efficiency improvement and learning practice. The retrospective and its participants are instead of trying to design and manage the environment unilaterally, investigating all the different angles and perspective the team can present and finding solutions to the problems existing within that environment. The joint team discussion and retrospective practices are evidence of this. 

Maximizing winning and minimizing loosing is not visual in the current retrospective practice. The teams are not afraid to use the retrospective as an arena for creating experiments on new practices where some might work and some might not. Team Delta and Echo, gave examples where the teams had tried to implement new practices and instead of going down with ship when the practice had not worked, leave ship and try something else. In Team Delta things that did not work for the members of the team simply did not get done and it was a joint understanding that actions that did not get any attention were bad practices. In Team Echo the team was not satisfied with current work practices and changed them drastically. After some months time they found that these new practices only made thing worse and instead of claiming ownership for the task and try to force it through the team decided to try something new. 

Through retrospectives we have seen that the participants are allowed to express their feelings regardless of their good or bad. This effectively counter the third governing value which is minimizing generating or expressing negative feelings. Team Echo and Delta both described events that showed participants raising negative issues and feelings towards the team. From the retrospective analysis for Team Zulu we saw that 89.3\% of the actions created came from negative issues. Even though allowing negative feelings are good thing some of the interview subjects said it could become a little bit to much negative sometimes, and that the retrospective is not a arena to vent. As was the case with Team Zulu the third and fourth feedback session revealed that the team had become more aware of raising also good feelings and issues during the retrospective and such they felt the retrospective had improved. This indicates that allowing negative feelings is important as one can learn and improve from them. However one should also ensure that good feelings are raised as they also provides the same opportunities and creates a more enthusiastic feeling about the retrospective.

As feelings are encouraged to share during the retrospective, to some moderation, the fourth governing value being rational also seemed to not be the apparent in the retrospective. Being rational implies censoring feelings and as we described above retrospectives encourages sharing of feelings towards the group. 

The consequences resulting from Model I governing values also seems to be rarely encountered in retrospectives in terms of the behavior. We have witnessed one case where an actor acted defensibly and such created an atmosphere that suppressed the other participants feelings. Creating what was described in an elephant in the room. It was also confirmed by the interviews that personnel issues were not addressed during the retrospective and such events could decrease the value gained from the retrospective. Other than this we have neither seen or heard about teams having defensive norms, or having defensive interpersonal relationships. 

Most of the learning consequences seems to be opposite of what is expected by Model I except single-loop learning and possibly decreased long-term effectiveness. It might not be that surprising that as the governing values are rarely encountered that the consequences of them is not either. Neither self-sealing, lack of public testing of theory or too much testing of theory in private seems to be appearing in teams conducting retrospectives. What is more surprising is that single-loop is quite occurring. We will dwell deeper into this in \autoref{discussion:learning-types}. In terms of decreased effectiveness we have no way compare these to any of our results as we lack a control group applying most of the governing values from Model I. 

The governing values from Argyris and Schön's Model I seems to not occur or be the system employed by agile development teams performing retrospective systems today. We have only seen two cases where the governing values of Model I has had any implications on the team. The first one was that of face-saving from one of the team members resulting in an atmosphere suppressing the other members feelings. This member later left the team. The other implications of Model I governing values is that of single-loop learning which most teams experience and will be discussed further in \autoref{discussion:learning-types}

\subsection{Learning Types}
\label{discussion:learning-types}
\subsection{Impediments for Learning}
% section organizational_learning (end)
\chapter{Results}
In this section we will report the results of our research. We will describe our results from the depth and breadth study, and the chapter is divided into two sections. The first section is the results that relate to the characteristics of the retrospective. This includes the output, the processes used in practice and the impediments that face the retrospective. The second section is all the results related to organizational learning.

\section{Retrospective Characteristics}
In this section we will first describe some key characteristics found during our depth study. We will then continue describing the output, processes and impediments found in both of our studies. 

\subsection{Key Characteristics}
From our depth study of team Zulu we found some key numbers and these numbers can also be seen in \autoref{table:key-numbers}.The retrospective reports spanned over a period of five years from August 2009 to November 2014. This amounts to 278 weeks and we are going to refer to week numbers from the first retrospective for the remainder of this report. 
\begin{table}[!h]
	\begin{center}
	\caption{Some key numbers from the retrospectives}
	\label{table:key-numbers}
	\makebox[\textwidth]{%
		\begin{tabular}{ l | p{0.5\textwidth}}
		\hline
		Key-value & Value  \\
		\hline
		Retrospective report period & 278 Weeks \\
		Number of total actions & 343 \\
		Number of unresolved actions & 65 \\
		Average actions per week & 1.23 \\
		Average unresolved action per week & 0.23 \\
		\hline
		\end{tabular}
	}
\end{center}
\end{table}
During the 278 weeks 77 retrospectives were held and and within these 343 actions were created, where 65 of these actions were still unresolved. This yields an average of 4.45 actions per retrospective and 0.84 unresolved per retrospective. This is equal to 1.23 actions per week, where one has 0.23 unresolved actions per week. 

In \autoref{figure:key-numbers} we can see the development of the numbers of actions. We can see that the team has had a pretty steady amount of actions with no abnormal spikes or changes until week 180. The total amount of actions follow the average expected actions quite close and reveals the steady number of actions.
At week 180 a slight increase in the number of actions begin. This lasts until until 237 when the amount of actions per retrospective starts to even out. 
The amount of active (or unresolved) actions has a steady amount of total actions increasing at about the same rate as the total number of actions. 

\begin{figure}[!h]
	\centering
	\includegraphics[width=\textwidth, keepaspectratio]{figures/key-numbers.png}
	\caption{A visual representation of some of the key numbers.}
	\label{figure:key-numbers}
\end{figure}
\afterpage{\clearpage}

When presenting the key-numbers, during the first feedback session, team Zulu mostly found our results agreeable with their own thoughts. The only surprise to the team was the amount of active actions. Their surprise came as they believed the number of active actions to be higher. The reason for this belief was that they thought they were worse at closing actions as the list of active actions seemed so long, but compared to the amount of total actions it seemed more reasonable. However it was pointed out that existing actions hinder new actions relating to the same problem to be created and thus the amount of active actions might not be accurate with how the team works with the problems, as an action might not be documented, but worked on never the less. 

When we raised the question on why the decrease of actions after week 237 the team gave their thoughts. During this period, they had acquired a new foreign developer within the team. Unfortunately the new developer had not lived up to the task, creating what the team called an ``Elephant in the room''. We will discuss this further in \autoref{results-elephant-in-the-room} as this is regarded as an impediment. 

\subsection{Output}
The output of the retrospective practice can be divided into several themes. We divide it into topics discussed, improvements and enthusiasm. 
The results in ``Topics Discussed'' describes our findings on which topics are discussed during the retrospective. The ``Improvements'' results describe how decisions are made and how the teams implement them. ``Enthusiasm'' describes how enthusiasm inflicts the retrospective practice. 

\subsubsection{Topics Discussed}
From both our case-studies we got an indication of what issues were brought to the retrospective. From our breadth study we learned which areas topics originated from. From our depth study we found which topics from our categories, described in \autoref{analysis-method}, were mostly influencing the retrospective. Below we will first describe the general results and then follow with the results from the depth study describing each category-set and some of the trends we identified. 

\paragraph{General}
In general we found three main areas discussed in the teams participating in our breadth study. The three areas were work environment, process improvement and technical issues. 

Work environment issues were described mainly by team Echo. They used the retrospective to improve or fix their work environment discussing issues such as noise in the working areas and bad wifi connection. We also saw some similar issues in team Zulu, however it was not their main focus. 

Process improvement was the second area we found. This was the main focus of almost all of the teams. Team Alfa, Charlie, Delta, Golf and Zulu had this as their main focus. Team Bravo had it as a main focus along with technical and team Echo discussed such issues, but rarely. Process improvement issues were regarded by most of the teams as the most valuable output from the retrospectives. Issues that were discussed included how the team communicated and how development should be done in a process perspective. 

The third area discussed during the retrospective was technical issues. Team Bravo had this as their second focus along with process improvement. It occurred in the other teams as well. In some teams, Alfa and Delta the issues were actively censored during the retrospective as they wanted to have a focus only on process improvement. Technical issues were usually related to the product and ways to either fix faults or improve the quality. 

\begin{table}[!h]
	\begin{center}
	\caption{Discussion Areas}
	\label{table:discussion-area}
	\makebox[\textwidth]{%
		\begin{tabular}{l p{0.5\textwidth}}
		\hline
		\textit{Discussion Area} & \textit{Team Focus}\\
		\hline
		Work Environment & Team Echo \\
		Process Improvement & Team Alfa, Team Bravo, Team Charlie, Team Delta, Team Golf, Team Zulu \\
		Technical & Team Bravo \\
		\hline
		\end{tabular}
	}
	\end{center}
\end{table}

\paragraph{Nature}
The retrospective analysis of team Zulu revealed that most actions are created as a result of negative problems that has occurred during the development. 89.3\% of the actions were negative, while 5.5\% of the actions were positive and acknowledged good working practices that would be continued. 5.2\% of the actions we lacked the context to determine whether they were positive or negative. As for the distribution of the actions over each retrospective there were no abnormalities except week 97 where there was an unusual amount of positive actions. However while looking into this week we found nothing in particular that could be identified as cause for this spike. As can be seen in \autoref{table:nature-results} the classification of the active actions pretty much mirrored the results from the total actions. 

\begin{table}[!h]
	\begin{center}
	\caption{Analysis results from the content analysis for the nature of the action.}
	\label{table:nature-results}
	\makebox[\textwidth]{%
		\begin{tabular}{| l | l | l | l | l |}
		\hline
		Category & \multicolumn{2}{|c|}{All Actions} & \multicolumn{2}{|c|}{Active Actions}  \\
		\cline{2-5}
		& Number & Percentage & Number & Percentage \\	
		\hline
		Positive & 19 & 5.5\% & 1 & 1.6\% \\
		Negative & 310 & 89.3\% & 57 & 90.5\% \\
		Undefined & 18 & 5.2\% & 5 & 7.9\% \\
		\hline
		\end{tabular}
	}
	\end{center}
\end{table}

\begin{sidewaysfigure}[!h]
	\centering
	\includegraphics[width=\textwidth, keepaspectratio]{figures/nature-l.png}
	\caption{The distribution of negative, positive and undefined actions across the timespan}
	\label{figure:nature-l}
\end{sidewaysfigure}
\afterpage{\clearpage}

The results from the nature context did not surprise the team at all. We asked if they would like the guess what the ratio between positive and negative would be and they said one-to-nine, which is quite close. Our results revealed five percent undefined and five percent positive with the remaining 90 percent negative. The team described themselves as problem-oriented and reasoned that this was the cause for the high amount of negative actions. They also said that they do, during the retrospectives, talk about good things that have happened, but as they are problem oriented it rarely gets documented. 

After reflection of the results through an internal retrospective the team decided to do a concerted effort to increase the positive attitude and number of positive issues during a retrospective meeting. One step taken was an informal attitude where team members tried to focus on positive events, and bringing them up during the retrospective meeting. Another step was the decision to add a numerator to track positives. Lastly they decided to add a ``positive/negative'' value to each action point. This in combination with the plan to implement a dashboard for data tracking let the team plot the positive/negative actions over time. During the fourth feedback session the team leader presented that they had already felt changes in the atmosphere in the retrospective meeting and described it as ``less of a funeral''. 

\paragraph{Context}
For the context of the actions analyzed, in the depth study, the majority were process related ones. The process actions numbered in 228, which is equal to 58.6\% of all the actions. The Technical ones numbered as 157 which is 40.4\%, while only 4 actions were undefined which results in 1\% of the total actions. As for the distribution over the timespan analyzed there where no abnormalities as can be seen in \autoref{figure:context-l}. For the active actions the results become more equal as seen in \autoref{table:context-results}. However it is worth mentioning that the active actions are a sub-group of the total and thus this result is probably a skewed grouping. 

\begin{table}[!h]
	\begin{center}
	\caption{Analysis results from the content analysis for the context of the action.}
	\label{table:context-results}
	\makebox[\textwidth]{%
		\begin{tabular}{| l | l | l | l | l |}
		\hline
		Category & \multicolumn{2}{|c|}{All Actions} & \multicolumn{2}{|c|}{Active Actions}  \\
		\cline{2-5}
		& Number & Percentage & Number & Percentage \\	
		\hline
		Technical & 157 & 40.4\% & 37 & 52.1\% \\
		Process & 228 & 58.6\% & 34 & 47.9\% \\
		Undefined & 4 & 1\% & 0 & 0\% \\
		\hline
		\end{tabular}
	}
	\end{center}
\end{table}

\begin{sidewaysfigure}[!h]
	\centering
	\includegraphics[width=\textwidth, keepaspectratio]{figures/context-l.png}
	\caption{The distribution of technical, process and undefined related actions across the timespan.}
	\label{figure:context-l}
\end{sidewaysfigure}
\afterpage{\clearpage}

\paragraph{Development Phase}
\label{section:development-phase}
Planning, testing, development, and documentation were the four dominant phases in which an action was related according to our content analysis of the retrospective reports. As can be seen from \autoref{table:development-results}. Planning being the biggest has a distribution value at 24.6\%. Second is the testing which 21.1\% of all the actions are related to. Development is related to 18.4\% and documentation is 13.2\%. Finally we have the remaining five categories Release, Build, Business Development, Bugfix and undefined which varies between 3.7-6.4 percent as can be seen in \autoref{table:development-results}. 
For the distribution of the different categories over time all of the categories are evenly distributed, in other words; No category is clustered to a specific period in time, but rather occurs evenly through the whole timespan. This can be seen in \autoref{figure:development-l}.
As have been the cases with the other themes the sub-group of the active actions mirrors the total actions with only minor variances as can be seen in \autoref{table:development-results}.

\begin{table}[!h]
	\begin{center}
	\caption{Results from the content analysis in which development phase the action regards.}
	\label{table:development-results}
	\makebox[\textwidth]{%
		\begin{tabular}{| l | l | l | l | l |}
		\hline
		Category & \multicolumn{2}{|c|}{All Actions} & \multicolumn{2}{|c|}{Active Actions}  \\
		\cline{2-5}
		& Number & Percentage & Number & Percentage \\	
		\hline
		Development & 89 & 18.4\% & 11 & 13.1\% \\
		Testing & 102 & 21.1\% & 18 & 21.4\% \\
		Documentation & 64 & 13.2\% & 16 & 19\% \\
		Release & 18 & 3.7\% & 4 & 4.8\% \\
		Build & 23 & 4.8\% & 6 & 7.1\% \\
		Business Development & 18 & 3.7\% & 5 & 6\% \\
		Planning & 119 & 24.6\% & 19 & 22.6\% \\
		Bugfix & 20 & 4.1\% & 2 & 2.4\% \\
		Undefined & 31 & 6.4\% & 3 & 3.6\% \\
		\hline
		\end{tabular}
	}
	\end{center}
\end{table}

\begin{sidewaysfigure}[!h]
	\centering
	\includegraphics[width=\textwidth, keepaspectratio]{figures/development-l.png}
	\caption{Timeline showing the distribution of the different development phases over time.}
	\label{figure:development-l}
\end{sidewaysfigure}
\afterpage{\clearpage}

The results from the retrospective analysis were quite different from the team expectation in the context of which development phase the actions were related. The team expected that build and documentation would be the two biggest groups. The content analysis revealed that planning was the biggest followed by testing, development and documentation. The team thought this made sense as most of the planning is directed towards process improvement and they do focus on process during the retrospective. The team speculated if the releases might have a correlation with planning actions coming right after. It was also again mentioned that actions might not be created even though they were discussed as actions already existed and that they had not been completed. 

The team expected build to be the phase that occurred most. It was a surprise that it was not better represented within the actions as it felt that it was discussed often in retrospectives. One of the members said that the lack of actions specific for builds might be the reason for why it was a problem. Even though it was discussed during the retrospectives.  

We asked the team if there were any categories they missed from the development phase analysis and two were mentioned. The first one was hotfix which the team often mentioned during the feedback session. The second one was flow as to how the development of the product progressed. For future similar analyses this should be taken into consideration. 

After the first feedback session with team Zulu we took the time to create a chart showing releases against the number of actions and planning actions as displayed in \autoref{figure:releases-and-planning}. These results provided interesting data. As can be seen in \autoref{figure:releases-and-planning} the releases does not have any impact on either the amount of planning actions or number of actions in total. This is interesting as the team themselves expected there to be a correlation.During the third feedback session the team leader also found this interesting. However he did not have any arguments for why this could be the case.

\begin{sidewaysfigure}[!h]
	\centering
	\includegraphics[width=\textwidth, keepaspectratio]{figures/releases-and-planning.png}
	\caption{Correlation between release and planning actions}
	\label{figure:releases-and-planning}
\end{sidewaysfigure}
\afterpage{\clearpage}

\paragraph{Collaboration}
The depth study of team Zulu showed that 45.3\% of the actions were undefinable in terms of the collaboration and the categories we had created for it. From the actions that were definable Communication was the biggest with 35.2\%. The second was external relations at 11.5\% and third competence at 6.9\%. Finally leadership was the smallest at 1.1\% of the total actions. The statistics can be seen in \autoref{table:collaboration-results}. For the distribution of the different categories over time, \autoref{figure-management-l}, most of the categories was evenly spread across the whole timespan. the only exception to this is week 146 where there is a clear spike of external relations. This spike was a result of the team attending a networking meeting in which they did a retrospective to better prepare them for the next networking meeting. This anomaly will be disregarded further in the report. 
The active actions shows that the three categories communication, competence and external relations evens out while leadership and undefined remains nearly the same with only some small variances. However it is worth mentioning that the active actions are a sub-group of the total and thus this result is probably a skewed grouping. 

\begin{table}[!h]
	\begin{center}
	\caption{Results from the content analysis regarding the collaboration influences of an action.}
	\label{table:collaboration-results}
	\makebox[\textwidth]{%
		\begin{tabular}{| l | l | l | l | l |}
		\hline
		Category & \multicolumn{2}{|c|}{All Actions} & \multicolumn{2}{|c|}{Active Actions}  \\
		\cline{2-5}
		& Number & Percentage & Number & Percentage \\	
		\hline
		Communication & 128 & 35.2\% & 14 & 20.9\% \\
		Leadership & 4 & 1.1\% & 1 & 1.5\% \\
		Competence & 25 & 6.9\% & 1 & 11.9\% \\
		External relations & 42 & 11.5\% & 11 & 16.4\% \\
		Undefined & 185 & 45.3\% & 33 & 48.3\% \\
		\hline
		\end{tabular}
	}
	\end{center}
\end{table}

\begin{sidewaysfigure}[!h]
	\centering
	\includegraphics[width=\textwidth, keepaspectratio]{figures/management-l.png}
	\caption{Timeline showing the distribution of the different collaboration categories over time.}
	\label{figure-management-l}
\end{sidewaysfigure}
\afterpage{\clearpage}

During the first feedback session the team expected communication to be the category with the most actions. If you disregard the undefined actions which was the biggest in terms of collaboration the team's expectation was correct. We asked if they could explain what kind of communication that were discussed most during the retrospectives. They said that communication between the different stages of the development and more oral communication rather than written were the kind of communications that were discussed the most. As for the other categories the team had no special feedback.

\paragraph{Trends}
While conducting our retrospective analysis of team Zulu, we uncovered some trends. By trend we mean actions that are related to the same issue/theme. The trends were identified by comparing actions towards each other noting which actions were similar to each other. Also noted were actions that still were not implemented. Interestingly the three trends we found turned out to have very few unimplemented actions. Which can indicate that the team has a focus on trends. We identified three trends. Bugfix, Scenario Template and Developer-Tester Communication. We'll go through each of these in the following sub-sections. 

\subparagraph{Bugfix}
\label{section:bugfix}
The first trend we recognized performing our content analysis was bugfixing. Developing computer systems is sure to create bugs and fixing them then becomes a natural part of developing software. In total we found 20 actions that were related to bugfixing. Of these, two were purely technical actions, five were technical and process related and the remaining 13 action were purely process related. Of the 20 actions nine were related to communication between team members. In \autoref{figure:bugfix} one can see that the total amount of bugfixing actions have increased steadily throughout the timespan. 

\begin{figure}[!h]
	\centering
	\includegraphics[width=\textwidth, keepaspectratio]{figures/bugfix.png}
	\caption{The total amount of bugfix actions over time.}
	\label{figure:bugfix}
\end{figure}

When we presented the bugfix trend to the team, during the first feedback session, they were surprised that we had identified it as a trend. The surprise came as a result that they had found a way to work with the bugs. Having dedicated bug-days was a practice that had been implemented and used for the last year which the team found very helpful. As can been seen from \autoref{figure:bugfix} the amount of actions still continue to increase after the implementation of the bug-day. When inquired by this the team told us that even though the bug-day worked very well, they still wanted to improve and that was why the actions still were increasing. 

\subparagraph{Scenario Template}\label{results-ca-scenario-template}
The second trend we discovered were in relation to a worktool called scenario template that team used to help specify requirements, create user stories and etc. In total we found 25 actions that were related to the scenario template. Of these 25 actions four of the actions were technical and process related, six were purely technical adjustments of the tool and 16 of the actions were process oriented on how the scenario template should be used. 18 of the actions were single-loop, only changing the effects which the scenario template provided. There were also six double-loop actions acknowledging root-cause issues with using the scenario-template and changes to reflect them. 
In \autoref{figure:scenario-template} the total amount of scenario template actions are shown over the 272 week long timespan. One can see that until week 163 the team has a slow increase in the number of scenario template actions. After week 163 however, a huge increase in number of scenario template actions occur. This continues until week 235, with a little slow period between week 180 and 205. At week 235 the team planned a meeting to go through the complete scenario template and after week 235 there are no more actions related to the scenario template. 

The trend of scenario template changes provided surprises both for the team and the researchers during the first feedback session. As described above the retrospective reports indicated that during week 235 a meeting were scheduled to go through the whole template. After this action no more actions occurred during the timespan. The researchers expected it to be a classic case of taking the root-cause, double-loop learning and the problem disappeared. However the scheduled meeting was never held. The team discussed several reasons for why this could be the case. One were that they still might be waiting to hold that meeting. Another was that a cleanup had been done, and changes to the scenario template still occurred, but that actions on it were not created. A third possibility was that the same period they had a team member that didn't work out as described in \autoref{results-elephant-in-the-room}. Unfortunately the discussion ended only with possibilities and no conclusion. 

\begin{figure}[!h]
	\centering
	\includegraphics[width=\textwidth, keepaspectratio]{figures/Scenario-tpl.png}
	\caption{The total amount of scenario template actions over time.}
	\label{figure:scenario-template}
\end{figure}
\afterpage{\clearpage}

\subparagraph{Developer-Tester Communication}
The final trend we observed during our content analysis was the communication between developers and testers. In total 25 actions were related to this. Of these 24 were process oriented and all the actions occurred from issues with a negative nature. 
\autoref{figure:dev-test-com} shows the distribution of the 25 actions over the 272 week long timespan. It can be seen that for the first 209 weeks the amount of actions increase slowly with only two to seven actions every 50th week. After week 209 however we see a dramatic increase in the amount of actions, before it completely stops in week 238. We were not able to find any possible reasons for this sudden stop from reading through the reports. However as can be seen from \autoref{figure:dev-test-com} there has been periods between actions as long as 45 weeks so it is possible that this stop can be such a break.  

\begin{figure}[!h]
	\centering
	\includegraphics[width=\textwidth, keepaspectratio]{figures/devtestcom.png}
	\caption{The total amount of developer-tester communication related actions shown over time.}
	\label{figure:dev-test-com}
\end{figure}
\afterpage{\clearpage}

The final trend, developer and tester communication returned similar feedback as the scenario template trend. The pattern identified in the content analysis where after week 238 the actions related to developer tester communication stopped. The team again found several possible solutions. The first one was, as described in \autoref{results-elephant-in-the-room}, the team member which did not fit the team. Another possible reason was secretary changes as well as scrum master's leave of absence. The team also said that most of the communication had been mostly oral and handover meetings had been introduced and that this worked pretty well. The team seemed to believe that all the mentioned reasons would explain the stop of developer-tester communication actions. 

\subsubsection{Improvements}
Which improvements are returned by the retrospective practice and how they are implemented were some of research questions we asked. From both of our case-studies we got some results relating to this. Our depth study gave us an insight into which kind of decisions are made during the retrospective. From both studies we got insight into how improvements were implemented. From the fourth feedback session with team Zulu we also learned that the team wished to develop a tool help them support implementation. We will describe each these results below.

\paragraph{Decision Making in Team Zulu}
\label{section:decision-making-zulu}
From the depth study of team Zulu we got insights into what kind of decisions were made during the retrospective. From the retrospective analysis the decision making results showed that the operational decisions occurred most in the actions as can be seen in \autoref{table:decision-making-results}. Operational decisions occurred in 53.2\% of the actions, while tactical was at 25.9\% and strategic was at 16.1\% of the actions. There were only four cases where we were not able to determine which kinds of decision making type it belonged to. For the distribution over time, as shown in \autoref{figure:decision-l}, there was no emerging patterns and all the decision making types were evenly distributed. The active actions mirrored the total actions almost equal as can be seen in \autoref{table:decision-making-results}.

\begin{table}[!h]
	\begin{center}
	\caption{Analysis results from the content analysis for the decision making perspective of the action.}
	\label{table:decision-making-results}
	\makebox[\textwidth]{%
		\begin{tabular}{| l | l | l | l | l |}
		\hline
		Category & \multicolumn{2}{|c|}{All Actions} & \multicolumn{2}{|c|}{Active Actions}  \\
		\cline{2-5}
		& Number & Percentage & Number & Percentage \\	
		\hline
		Strategic & 55 & 16\% & 10 & 16.1\% \\
		Tactical & 89 & 25.9\% & 18 & 29\% \\
		Operational & 195 & 56.9\% & 33 & 53.2\% \\
		Undefined & 4 & 1.2\% & 1 & 1.6\% \\
		\hline
		\end{tabular}
	}
	\end{center}
\end{table}

\begin{sidewaysfigure}[!h]
	\centering
	\includegraphics[width=\textwidth, keepaspectratio]{figures/decision-l.png}
	\caption{A timeline showing the distribution of the different decision making decisions for all the actions.}
	\label{figure:decision-l}
\end{sidewaysfigure}
\afterpage{\clearpage}

The team thought that it was a good sign that the team was able to do strategic decision making. When we presented the results for the decision making analysis, the team was pleased to see that they had all of the three categories represented. They were especially pleased to see that there was a substantial portion of strategic actions. We asked what could be the reason for this and the team responded that they felt autonomous. As they described it, the company allowed the different development teams to have a fairly large amount of independence allowing room for strategic choices within the team. 

\paragraph{Implementation}
\label{question-6}
The steps taken to implement actions created during the retrospective and whether unresolved actions were a problem varied between the interviewed teams. 

Team Echo used several techniques to enforce actions. Some actions were added to the sprint backlog as it helped reminding the team that they needed to be done. The actions that were related to the work environment was usually handled by the SCRUM master. Reaching a common consensus was something that the team regarded as a beneficial way of getting things done. Finally the team, along with other interviewed teams, always assigned a name to the actions found during the retrospective. This ensured that a person would have a responsibility to the action and at every retrospective, the team would take a quick round to see if the action had been completed. We asked the developer what happened if someone forgot to do it and he replied that it was noticed by the other members of the team and it had never occurred that an action had been unresolved over two retrospectives. This resulted in that most actions became resolved as was the case with other teams that assigned names to the actions. 

The SCRUM master of team Zulu described the current action assignment as ``push'' based where an issue was delegated to a team member, often the one with expertise relevant to the issue. Also mentioned was a desire to introduce a more ``pull'' based system where tasks could be chosen by team members at their leisure. A major issue was the problem of enforcing process related actions, as there existed no formal tool or process for ensuring that the process actions were actually employed. A lot of the actions were quick fixes that were handled rapidly by whoever were delegated the task.

Team Delta said that most actions were resolved. However new routines that were part of an action that was not working would not be done regardless of names assigned. These new routines however was later deemed bad routines as it did not work when it was practiced. 

Team Alfa managed to resolve most of the actions created. They assigned names to the different actions and the project leader said this was more effective rather than not doing it, which they had done earlier. Some actions however the team wasn't able to resolve. There were several reasons for this. Some of the actions became really creative and thus were difficult to implement. Other actions were huge and required resources the team didn't have. 

Follow-up by the SCRUM master of the team was generally regarded as an important measure to help resolve actions from the retrospective by several teams. Also adding the actions to a SCRUM board or a separate board was regarded as a good measure to get actions resolved. 

The SCRUM master of team Charlie admitted that very few of the actions they created were resolved. He told us that actions were not assigned to individuals, but rather the group as a whole. As part of the leadership for the whole department he had participated in a retrospective for the department as a whole, where individuals had been assigned to different actions. This had not worked in that retrospective and thus he was hesitant to do this in his own team. He hoped that having a white board to put the actions up on would help remind the team to resolve some of the actions. 

\begin{table}[!h]
	\begin{center}
	\caption{Action Follow-Up Techniques Used}
	\label{table:follow-up-techique}
	\makebox[\textwidth]{%
		\begin{tabular}{ l | l | l }
		Team & Follow-up technique & Satisfied \\	
		\hline
		Echo, Delta, Alfa, Foxtrot, Zulu & Assigning Name to action & Yes \\
		Charlie, Bravo & Assigning to group & No \\
		Foxtrot & Adding to backlog & No \\
		Delta & Visualizes Action & Yes \\
		Echo, Golf & Handled by SCRUM-master & Yes \\
		\hline
		\end{tabular}
	}
	\end{center}
\end{table}

\label{question-21}
Major breakthroughs that stemmed from retrospective all came from process related improvements. Team Delta described how one team member had brought up a personal issue, the team member felt he did not get enough help from the other team members. The team was then forced to examine how this had occurred, and how what they could do to prevent it. This issue was resolved. This was described as a ``tough retrospective'', but the team was very happy with this issue being raised, and it was regarded as one of the best retrospectives they ever had, because it worked directly on team dynamics. The scrum master for team Delta said:

\begin{quote}
We were all very pleased with this issue being raised, it was one of the best retrospectives we ever had, because it directly addressed team dynamics, we handled it pretty well, everyone were eager to discover the problems and how they could help.
\end{quote}

The team Delta SCRUM master described this issue as a ``heavy'' issue to raise, especially for the person who was not getting help. The SCRUM master considered this a proof that the team trusted each other, and said that the issue would never have been raised in a team without trust.

\paragraph{Decision to make a dashboard}
\label{dashboard}
Team Zulu held a retrospective session where the impressions from the analysis and earlier feedback sessions were discussed. The authors of this report were not present, but received the results through a talk with the team leader and the team SCRUM master, as well as a written report that is further described in \autoref{result:learning-reflection-zulu}. One of the actions they wished to implement, as a result of the depth study, were to develop a dashboard to help coordinate and focus the implementation of actions. This dashboard would be based on the method and form of the analysis done by the authors. The team decided that the team should at least contain the data seen in \autoref{list:dashboard-data}. This list was based on a similar list provided in the written feedback from the team. 

\begin{enumerate}
\label{list:dashboard-data}
\item Number of actions over time.
\item Number of positives over time.
\item Distribution of positive and negative actions over time.
\item Distribution of single- and double-loop actions over time.
\item Open and closed actions over time.
\end{enumerate}

We earlier described how team Delta visualized their actions and felt that it helped implement actions. As the wish to visualize is apparent in both team Zulu and team Charlie suggests that visualization could be a good strategy to help follow-up the implementation of actions. 

\subsubsection{Enthusiasm}
Our case-studies revealed that enthusiasm both inflicted the retrospective and was affected by it. The results revealed that enthusiasm could create both a positive loop and a negative loop. They also revealed some factors that affected the enthusiasm, being oversight and ownership and trust. We will describe each below. 

\paragraph{Positive Loop}
Described by several teams, Alfa, Delta, Echo and Zulu, enthusiasm was able to be a part of a positive loop related to the retrospective. The loop could be described as the following:

\begin{quote}
\textit{If the retrospective produced any implementable actions and those actions were implemented it would produce more enthusiasm for the retrospective practice and therefore increase the chance of future actions actually being implemented.}
\end{quote}

It was emphasized that change was important for the retrospective practice by all the teams where the subject came up. It would produce enthusiasm and help improve the working practices. 

\paragraph{Negative Loop}
As two sides of a coin enthusiasm could, in addition to create a positive loop, create a negative loop. The SCRUM master of team Charlie told us that unresolved actions could create a negative loop, where enthusiasm for the retrospective went down as improvements never came and the low enthusiasm made sure fewer actions were implemented. And that it could be a challenge. 

\paragraph{Oversight}
From the second feedback session with team Zulu during the depth analysis we learned that oversight over implementation rate of actions had affected the retrospective. The team had not previously appreciated just how much had been accomplished, and that this had lead to a lower enthusiasm around the retrospective. When the team was confronted with the completion rate they were surprised and pleased with the amount they had accomplished. In order to continue this more objective oversight the team decided to create a dashboard in order to have better view of the statistics concerning implemented versus unimplemented actions as described in \autoref{dashboard}. 

\paragraph{Ownership and Trust}
The SCRUM master of team Zulu also had some reflection on how the team influenced the enthusiasm. During the second feedback session the SCRUM master spoke about how he considered his team mature and willing to learn. Over time the team had developed an ownership feeling regarding the retrospective, and that this had been developed through working with it over time. Another important element was that a high degree of trust between team members led to productive sessions. An important part of this is that every team member feels that they are taken seriously. This helped create positive enthusiasm for the retrospective practice. All of the other teams we spoke to about the subject confirmed that ownership toward the development process helped create enthusiasm for the retrospective practice. 

\subsection{Processes}
In this section we will detail the results we discovered during our studies. The results will be divided into before, during and after the retrospective meeting.

\subsubsection{Before the retrospective}
This subsection will contain our observations on work done before the retrospective. Specifically the considerations ``Preparations'', ``External Facilitator'' and ``Encouraging through external things'' will be described.

\paragraph{Preparations}
When asked about how the team members prepared for a retrospective the team Zulu SCRUM master commented that there was a variety of approaches, some team members made lists beforehand and arrived prepared, while others were more impulsive and decided on their issues during the retrospective.

\label{question-12}
Of all the teams we talked with no one except one team used any external or specific tools to gather information and prepare the retrospective. The one team that used some other information than what was gathered at the retrospective used lead times as a source of information. 

\paragraph{External Facilitator}
There were different views on facilitating retrospectives. Several teams used an external facilitator and said that they would encourage others to do the same. The benefits was that the external facilitator was able to see things that existed within the team, that the team themselves were not aware of. The external facilitator was not hired as a facilitator, but rather a SCRUM master from another development team.

The use of external facilitators was an interesting concept to team Zulu, and the one experience they had with using one had been a positive experience. When the team Zulu SCRUM master was asked about if he felt like a leader, or if the team viewed him as a leader, the SCRUM master said he felt more like a facilitator, and that he didn't think the team considered him a leader. Though he was mindful of this possibility. When asked about the inclusion of the project leader he considered as long as the team was not afraid to speak their minds this could make it easier to make strategic decisions.

Other teams had not been using external facilitators. They used their regular SCRUM master. Common for the SCRUM masters were that they all felt like they were a facilitator rather than a leader during the retrospectives. Those we spoke to about external facilitators were positive to the idea and mentioned they might want to try it out in the future. 

\begin{table}[!h]
	\begin{center}
	\caption{Usage of external facilitator}
	\label{table:external-facilitator}
	\makebox[\textwidth]{%
		\begin{tabular}{ l | p{0.3\textwidth}}
			External facilitator & Teams \\
			\hline
			Used external facilitator & Alfa, Echo \\
			No external facilitator & Charlie, Delta, Foxtrot, Golf, Bravo \\
		\end{tabular}
	}
	\end{center}
\end{table}

\paragraph{Encouraging Through External Benefits}
\label{question-9}
Encouraging learning with external benefits like bonuses and such was not used by any teams. The closest was some teams that had sometimes brought some pastries to the meeting. One developer said that encouraging through bonuses would be a destructive force in the retrospective and the focus would be removed from the process.


\label{question-3}

\subsubsection{During the retrospective}
From both our case-studies we got insight into how retrospective meetings were conducted in todays practice. We will describe these results below and it is split into three parts. The first part is which practices are used by the teams today. The second part is the frequency and duration for the retrospective meeting for all the teams in the case study. The final part is reflecting on the team factors that affect the retrospective meeting.

\paragraph{Retrospective Practices Used}
\label{section:retrospective-practices-used}
When we asked the different developers and SCRUM masters how they conducted the retrospectives we got a wide range of answers. 

The current state of the retrospective in team Zulu was described as a simple meeting where team members could speak their mind and discuss any issues they themselves wanted to bring up. However they had experimented with some different retrospective techniques and found them interesting, and considered using them occasionally. An important part of the retrospective was an attempt to reach a consensus. The tone of the retrospective was described as light, but sometimes could get more heated during discussions. The retrospectives were held every 3rd week, but could sometimes be moved due to pressure related to releases or other pressing issues.

Team Echo started their retrospective with ranging on the range one-to-nine the three categories; team flow, team moral and technical quality. This allowed them to see trends on the three categories. When we asked what happened if something was graded really low the response was that it had not happened yet. We also asked if the opposite, what happened if something was graded really high and the response was the same that it had not happened yet. After the grading the team used KJ-sessions, where one write positive and negative issues on post-it notes and then discuss them in the group. Which Dingsøyr et al. describes as a structured brainstorming technique \cite{Dingsoyr2003}. According to the developer the things that came up during the KJ-session were really specific things related to the work environment. The reason for this was that the team worked on different pieces of software and that each sub-team had their own process. 

Team Alfa described how KJ-sessions functioned as a good method for engaging team members. Members that were passive or silent during discussions could be an obstacle to learning if their feedback weren't gathered during the retrospective. The KJ-session helped hinder this as all the participants were forced to provide some feedback to the meeting.

Two of the teams, Charlie and Delta, varied each time how they conducted the retrospectives. They followed the five steps devised by Derby and Larsen \cite{Larsen2006}, which is described in \autoref{section:Derby-Larsen-Structure} and varied the techniques for gathering data and generating insights. The reasons for this was to challenge the comfort zone of the members as well as counter group thinking and monotone activities. Group thinking is the desire for cohesion in a team and McAvoy and Butler \cite{Mcavoy2007} refers to it being considered as way of ineffective decision-making. 

We asked the SCRUM master in team Delta if it always was new techniques or if some were repeated. The response was that the same technique never came twice in a row, but techniques that worked well would be repeated on several retrospectives. As to how they found the new techniques they used several sources. Blogs, websites like retromat \cite{retromat2015} and agile podcasts were all mentioned as a good way of obtaining new techniques. 

During our discussion we asked if there were any downsides to using varying techniques. Team Charlie's SCRUM master said that that for some participants it could become a little bit too much. This could result in lower enthusiasm for the retrospective. Another downside was that for some techniques the focus could move to the technique instead of the issues which could lower the results of the retrospective. However he said that he still found that varying the retrospective techniques was beneficial to the retrospective. 

\begin{table}[!h]
	\begin{center}
	\caption{Retrospective Techniques Used}
	\label{table:retrospective-techique}
	\makebox[\textwidth]{%
		\begin{tabular}{ l  l }
			\hline
			Team & Technique \\
			\hline
			Echo, Alfa, Foxtrot & KJ-session \\
			Zulu Golf, Bravo & Team Discussion \\
			Alfa, Charlie, Delta & Varying Techniques \\
			Echo & Team Barometer \\
			Foxtrot & Weather Forecast \\
			\hline
		\end{tabular}
	}
	\end{center}
\end{table}

\label{question-3a}
\paragraph{Frequency and Duration of Retrospectives}
The duration and frequency varied between the different interview participants. An overview can be seen in \autoref{table:frequency-duration} Team Delta held retrospectives every week, while team Charlie held every second week and yet team Echo held every third week. The SCRUM master for the team Delta said that doing it every week gave continuous follow-up. The team regarded the retrospective as the most important meeting during the development and all the team members saw the value that the retrospective provided. 

Team Alfa performed retrospectives at irregular frequency. Instead of conducting retrospective after each sprint or at a given time they performed retrospectives after they had finished each major feature in their project. They also conducted retrospectives whenever one person in the team felt it was necessary. Usually a retrospective was held once a month. This resulted in two kinds of retrospectives being conducted. One feature retrospective where persons actively working on that feature participated and the complete development process of that feature was discussed. And one work process retrospective which handled general work processes for the team. This latter retrospective was conducted by an external facilitator. Of the two retrospectives the feature retrospective was the most common. 

Both the developer and project leader in team Alfa found that feature driven retrospectives worked very well. However the developer admitted that he sometimes missed having all members of the team participating in the retrospective. Currently only the people actively participating in the development of the feature was invited to the retrospective. The developer said that there was a risk that some members of the team might participate in a retrospective rarely as their assignments didn't necessarily result in a feature being created. 

There were varying degrees of participation to the retrospective, some teams like team Golf had every team member attend, while team Bravo only included testers in the release retrospectives. 

\label{question-3b}
The duration of the retrospectives varied between the teams, from a fixed amount of hours to until ``we are done''. For those having fixed time one team had 1 hour, while another had 2 hours long retrospectives. 

\begin{table}[!h]
	\begin{center}
	\caption{Frequency and duration for the different teams}
	\label{table:frequency-duration}
	\makebox[\textwidth]{%
		\begin{tabular}{ l | l | l }
		Team & Frequency & Duration \\	
		\hline
		Delta & Every week & 1 hour \\
		Charlie, Echo, Bravo, Golf & End of every sprint & 1-2 hours \\
		Foxtrot & Every second week up to six months & 0.5-4 hours \\
		Alfa & End of every released feature & 1-2 hours \\
		Alfa & On request by team member & 1-2 hours \\
		\hline
		\end{tabular}
	}
	\end{center}
\end{table}

\paragraph{Team Factors}
\label{question-20}
When asked about the impact team dynamics has on retrospectives most teams emphasized the need for a positive culture and an eager attitude. 

Team Echo described how one team member was a very ```negative'' person, constantly bringing up problems that needed to be fixed, however in the context of the retrospective this was seen as a positive, since it brought up necessary issues. 

Team Charlie was a small team of four developers, where all were highly experienced, with especially two of the developers holding very senior positions within the company. The two senior developers were described as especially strong personalities. The team was described as very eager to learn and always searching for ways and ideas that could be used to improve the team. The strong personalities influenced the retrospectives in that sometimes the junior developers would agree without arguing strongly for their own ideas or suggestions. The SCRUM master told us that this could be a challenge during the retrospective.

In team Alfa they described how some team members were very outspoken and others team members had a tendency to take issues being raised very personally, when these type of team members collided it resulted in friction during the discussion. 



\subsubsection{After the retrospective}
\label{section:after-the-retrospective}
Our discoveries concerning work done after the retrospective mostly concerns the steps taken to implement the decisions that were made as a result of the retrospective session. Mostly teams with a clear protocol for designating responsibility for the action implementation were satisfied with their implementation process. Specifically assigning a name to an action instead of to the group as a whole seemed to lead more satisfaction in relation to the action implementation. When asked about reflection of the retrospective process on a team level no teams practiced this, however the SCRUM masters at team Delta and team Charlie practiced this at SCRUM master meetings. A further description on the implementation of retrospective actions can be found in \autoref{question-6}.

\subsection{Impediments}
In this section we discuss the impediments we discovered during our work. These are divided into team factors and retrospective timespan.

\subsubsection{Team factors}
In this section we elaborate on our findings of impediments relating to team factors, these are divided into ``Cultural differences'', ``Team Changes'' and ``Team personalities''.

\paragraph{Cultural differences}
\label{results-elephant-in-the-room} 
As uncovered by our depth study of team Zulu they had a decrease in actions for a period of several months. They reasoned that having a foreign developer that created an ``Elephant in the room'' during the retrospectives was a possible reason for this. The situation absorbed the other problems within the team. No one would be rude and tell that the new developer that he was the problem. As one of the team members described during the feedback session: 
\begin{quote}
``We actually discussed it once at the coffee after lunch, the retrospectives at the moment were just a waste of time''. 
\end{quote}
When asked what was the reason for the developer not living up to his expectations the team told us that face-saving and cultural differences made it difficult to communicate properly and that tasks that were assigned to him weren't satisfiable. 

When this came up during our interview with the team Zulu SCRUM master he agreed with the observations done by the team and elaborated that the communication issues had caused multiple problems.  These communication problems became so prevalent that that they directly influenced retrospectives. Team members would consider the communication problems so major that they would consider more minor, and fixable issues irrelevant, thus leading to little or no issues being solved at all. This was not handled in the retrospective since personnel issues were not considered within the domain of the retrospective.

On the topic of norms or cultural differences having an impact the participants in the breadth-study reported low or no impact. team Golf reported some issues, but most of the work done on this area was done outside the retrospective. This work consisted of talking with team members ahead of the retrospective as well as building a culture ahead of issues arising. 


\paragraph{Team Changes}

For team Zulu the foreign developer quit the team after a period of 34 weeks in week 263. This still meant that there was a low action period of 15 weeks between week 263 and 278 and this period remained unexplained.When inquired about this the team had several possible reasons. One were that there were already to many active actions on the plate resulting in fewer getting made. Another were that the communication within the team improved after team had lost the developer creating the problems. A final possible reason was that the secretary for the retrospectives changed a lot within that period of time. 

After discussing the decrease in actions period the team mentioned that it would be interesting to see if there were any correlations between retrospective actions and team changes other than the one already explained. For the third feedback session we created such charts as can be seen in \autoref{figure:team-changes} and \autoref{figure:secretary-changes}. 

In \autoref{figure:team-changes} we can see that the period of low actions starting at week 237 and lasting out week 278. In that period the foreign developer, T7, joins and leaves the team. In the same period the SCRUM master of the team takes a leave of absence for a period of 20 weeks which also might have influenced the period. 

For the secretary changes, shown in \autoref{figure:secretary-changes} we can see that the changes in who writes the reports doesn't seem to influence the number of actions that comes out of the retrospectives. The only exception to this could be the period after week 263 where there are a lot of changes, as the team described. However we believe this to be unlikely as between week 270 and 275 the secretary stays the same and the amount of actions follows the trend of low number of actions and as changes earlier in the timespan haven't revealed any effects. 

\begin{sidewaysfigure}[!h]
	\centering
	\includegraphics[width=\textwidth, keepaspectratio]{figures/team-changes.png}
	\caption{Team changes and the amount of actions for the retrospectives across the 278 Weeks.}
	\label{figure:team-changes}
\end{sidewaysfigure}

\begin{sidewaysfigure}[!h]
	\centering
	\includegraphics[width=\textwidth, keepaspectratio]{figures/secretary-changes.png}
	\caption{Secretary changes and the amount of actions for the retrospectives across the 278 Weeks.}
	\label{figure:secretary-changes}
\end{sidewaysfigure}
\afterpage{\clearpage}

\paragraph{Team personalities}
The SCRUM-master of team Charlie said that personality might hinder the function of the retrospective. In his department SCRUM masters had been assigned to developers in teams where there only was developers. This had for some of the teams become a problem. In those teams that the SCRUM master had low enthusiasm for the retrospective practice, as they rather would just do regular developing, this low enthusiasm could spread to other team members.  Another incident where team personalities impacted the retrospective was in team Charlie, where two senior developers would dominate conversations, leading to the junior developers becoming timid and lowering their input during the retrospective.


\subsubsection{Retrospective timespan}
\label{section:Retrospective-timespan}
In team Foxtrot, a developer told us that the one hinder for getting value from the retrospective was too long timespan between the retrospectives. In that team retrospectives could be held on irregular times ranging from two weeks up to six months. If the timespan was to big, the retrospective returned no value as there simply was to much too discuss according to the developer. 


Team Alfa's feature driven retrospectives could lead to some developers being excluded for long periods of time as described in \autoref{question-3a}.

\subsubsection{Distributed Team}
Team Golf had parts of the team in Norway and parts of the team in China. This resulted in problems both at conducting the retrospective, due to timezone differences, and cultural differences. In order to ensure that everyone spoke up they had members submit written issues before the retrospective meetings and had a strong focus on the discussion and order of issues being resolved democratically. 

\begin{table}[!h]
	\begin{center}
	\caption{Value Decreasing Hinders for the Retrospective}
	\label{table:value-decreasing-hinders}
	\makebox[\textwidth]{%
		\begin{tabular}{ l | p{0.3\textwidth} | p{0.5\textwidth} }
		Team & Hinder & Description \\	
		\hline
		Alfa & Rare participation from some members & As the feature driven retrospectives only invited people who had worked on a specific feature, team members working on legacy code could have a long timespan between participation \\
		Charlie & Personality & SCRUM-masters who is not motivated to do retrospective creates low enthusiasm in the group. \\
		Foxtrot & Timespan between retrospectives &  Having too long time-spans between retrospectives creates more topics which takes longer to discuss and thus is unproductive. \\
		Golf & Distributed team and cultural differences & The team was distributed and had large cultural differences, this was countered by using written preparation and having a strong focus on democratic decision making. \\ 
		\hline
		\end{tabular}
	}
	\end{center}
\end{table}

\clearpage

\section{Organizational Learning in Retrospective Practice}
In this section we will look at the results related to learning and organizational learning. We will first describe some examples of learning that have occurred within some of the teams. Then we will describe which types of learning occurs. Then some results on how the teams reflect about their own learning. After that we will describe the learning obstacles identified. Finally we will give an overview of the learning enhancing factors and potential improvements that the teams identified. 

\subsection{Learning Through Retrospective Practice}
\label{question-11}
To create a common ground for discussing organizational learning we asked if there was any specific things that had changed in how the team worked or thought, through the retrospective. We will described three of these episodes here. 

One team, Delta, had for a while been using much time on estimating the time needed to complete a user story. The estimations turned out to be generally wrong and during a retrospective this had become a discussion. This resulted in the team changing the practice completely. Instead of using much time on estimating time they used little time on giving story points for the stories which gave them a better estimate on the workload required to complete a story. 

Team Echo had switched from SCRUM to KANBAN and back again over the course of half a year. This had been a result of having very time consuming process with a lot of steps from one begun planning until one user story was completed. The developer told us that at one point they had used more time on doing processes than developing. The team brought it up on the retrospective and they decided to try KANBAN instead. After a period of KANBAN the team found out that they needed some more structure than what KANBAN provided. This resulted in them reevaluating SCRUM and restructured it so it fitted what the team wanted better. 

An example from Team Zulu was the arrangement of a ``bug-crunch'' day that was created during the retrospective lead by a external facilitator. The ``bug-crunch'' day is a day set aside solely to the elimination of bugs and similar issues, the day is held at regular intervals. This was described as a very positive change that had led to a clear decline in bugs. 

\subsection{Types of Learning}
When we say types of learning we refer to the learning types single-, and double-loop learning. From our studies we gained some insights into which learning types occurred the most through the retrospective practice. We will first describe our results from the depth study and then the results from the breadth study. 

\subsubsection{Learning Types in Depth Study}
In terms of organizational learning each action in team Zulu's retrospective reports could be defined as single-loop, double-loop or undefined. The results yielded from the retrospective analysis showed that single-loop was the most occurring type of organizational learning with 66.4\% of the actions. Double-loop had 27.2\% of actions, and the rest was undefined at 6.4\%. The distribution over the timespan, \autoref{figure:learning-l} of the analysis showed that the three categories were evenly distributed. The active actions were very similar to the total amount of actions and only had some small negligible variances as can be seen in \autoref{table:organizational-learning-results}.

\begin{table}[!h]
	\begin{center}
	\caption{Results from the content analysis regarding the organizational learning nature of the action.}
	\label{table:organizational-learning-results}
	\makebox[\textwidth]{%
		\begin{tabular}{| l | l | l | l | l |}
		\hline
		Category & \multicolumn{2}{|c|}{All Actions} & \multicolumn{2}{|c|}{Active Actions}  \\
		\cline{2-5}
		& Number & Percentage & Number & Percentage \\	
		\hline
		Single-loop & 227 & 66.4\% & 41 & 66.1\% \\
		Double-loop & 93 & 27.2\% & 16 & 25.8\% \\
		Undefined & 22 & 6.4\% & 5 & 8.1\% \\
		\hline
		\end{tabular}
	}
	\end{center}
\end{table}

\begin{sidewaysfigure}[!h]
	\centering
	\includegraphics[width=\textwidth, keepaspectratio]{figures/learning-l.png}
	\caption{Timeline showing the distribution of learning loops for the total actions.}
	\label{figure:learning-l}
\end{sidewaysfigure}
\afterpage{\clearpage}

During the first feedback session with team Zulu the assumptions they had were as expected. They used the term direct causes as single-loop and root causes as double-loop, but we will continue using the terms of single-loop and double-loop. They expected that single-loop would have the most occurrences, even though they maintained a double-loop focus during the retrospectives. This expectation is reflected in our results quite well as 66 percent were single-loop and 25 percent double-loop. The reason they expected it to be more single-loop rather than double-loop was that it was easier to do single-loop and as one team member said: ``...sometimes things just need to be fixed''. 

The team were satisfied that they had what they considered a good ratio of double loop to single loop actions. The team made a formal decision to work on identifying opportunities for root cause and double loop actions instead of fixing symptoms. Another action step taken was to formally log actions with a single loop or double loop attribute, as seen in \autoref{list:dashboard-data}.

\subsubsection{Learning Types in Breadth Study}
\label{question-13} 
The teams we interviewed had varied focus on root-causes. Several teams admitted that they rarely tried to find the root causes and only fixed the effects and not the underlying problems, thereby only conducting single-loop learning. However they told us that it really was something they should do or that they wished to do. 

Team Charlie and Delta used the technique ``five times why'' to dig down to the root causes of the issues. They tried to find the root causes for every issue whether it was technical, procedural or personal. The SCRUM master presented us with a case where one of the developers had not been able to finish up their assignment. After digging into the problem they found the root cause which turned out to be that the person was not able to get help from the rest of the team. This was then addressed and actions were taken to solve it. The SCRUM master also told us that when technical issues did come up they dug down to see if there actually was a technical problem or a procedural problem. In most cases it turned out to be a procedural problem that was the root cause of the technical issue.

In team Alfa they simply discussed the issues until they found a concrete action that would hinder the issue to easily return at a later time. They started the retrospective with identifying issues. Then everyone would use three dots and place them on the issues which they found most important. Then they would start at the highest ranking issue and discuss it until they found the root cause and a way to counter it. Then they would continue on with the next issue. The developer in the team provided an example. During a retrospective a issue about long merging times had been brought up. After discussing the issue they found the root-causes for the long merging times were that they had too few releases as well as too big branches. Their solution to this was to start using another version control program making branches smaller in the transfer process as well as for the future, creating smaller tasks. 

One developer in team Foxtrot told us that some issues were able to be solved immediately within the meeting. Other more difficult issues could require more investigation and thus would be followed up in the next retrospective. 

\begin{table}[!h]
	\begin{center}
	\caption{Root-Cause identifying techniques used.}
	\label{table:root-cause-technique}
	\makebox[\textwidth]{%
		\begin{tabular}{ l | l | p{0.5\textwidth}}
		Team & Technique & Description \\	
		\hline
		Charlie, Delta & Five times Why & Through working in groups team members ask ``Why?'' five times in order to challenge their habitual thinking ~\cite{Larsen2006} \\
		Alfa & Team Discussion & A discussion in the team where one discusses the issue until one find an action such that the issue won't resurface. \\
		\end{tabular}
	}
	\end{center}
\end{table}

\subsection{Reflection About Learning}
Through our studies we investigated reflection on learning. The results we have split into two parts. The first one is ``Practices on Learning Reflection'' which describes the practices that the teams from our breadth study use to reflect on their own learning. ``Reflection on Learning Through Feedback Sessions'' is the second part which describes the reflections and learning that team Zulu did through the feedback sessions which we held. 

\subsubsection{Practices on Learning Reflection}
\label{question-15}
We asked whether any of the teams ever reflected on how they learned from the retrospective.

The SCRUM master of team Charlie said he was a part of the firm's community of practice on development process. In this community they reflected quite a lot about both how to conduct the retrospectives and the results of them. However the results from these reflections never made it outside the community and to the teams. When we pointed this out to the SCRUM master he realized that this should not be the case. Some teams suffered from low enthusiasm about the retrospective and if the reflections reached these members that problem could possibly be countered. Team Delta was also a part of this community of practice.

In team Alfa, during the work process retrospectives the project leader found that they reflected over learning as part of the discussion. A developer in the same team however found that the last year the reflection had decreased a little. They had earlier used reports from earlier retrospectives and used time on reflecting on how things from that retrospective were handled in terms of resource management as well as things that did not get handled. This had helped in the long run. 

Most teams however did not use any time on reflecting on how they used or learned from the retrospective. However when we asked the question, most seemed to realize that this would be a good way of increasing the value from the retrospective. 

\begin{table}[!h]
	\begin{center}
	\caption{Learning Reflection}
	\label{table:learning-reflection}
	\makebox[\textwidth]{%
		\begin{tabular}{p{0.5\textwidth}  l}
		\hline
		Technique & Team \\
		\hline
		Reflection in community of practice & Charlie, Delta \\
		Reflecting in discussion & Alfa \\
		Using Earlier Retrospective Reports & Alfa \\
		No reflection & Echo, Foxtrot, Bravo, Golf \\
		\hline
		\end{tabular}
	}
	\end{center}
\end{table}

\subsubsection{Reflection on Learning Through Feedback Sessions}
\label{result:learning-reflection-zulu}
From the first feedback session team Zulu reflected on the analysis conducted. The team found the presentation and analysis very useful. It was described as a refresh on retrospectives and they speculated if it was possible to develop a tool that could codify the actions from a retrospective and visualize them on a dashboard. They found the presentation provided a good awareness on the retrospectives, which earlier had not been that good. They also decided that they should focus more on finding root-causes as the presentation gave a short introduction to single-loop and double-loop learning.

During the second feedback session team Zulu's SCRUM master reflected on the learning results of the retrospective analysis. An interest was expressed in doing a more comprehensive analysis of the retrospectives within the team, but a major concern was a possible waste of time, with little return on investment. When discussing the process of spreading the lessons learned within the team to the outside there was no clear process for this, but creating a ``notable efforts'' documentation where positive efforts were documented was considered. 

The fourth feedback session consisted of a presentation from the team leader, a discussion along with the team leader and SCRUM master and a report from an internal retrospective held about the depth study by the team. 

``What have we learned from 77 retrospectives'' was the title of the presentation the team leader held. The results of the analysis and the impact it had on the team were presented. 

He talked about how retrospectives could be sad and described some of them as ``more depressing than a funeral''. He also described how it could be hard to handle personal issues and the tendency of issues to pile up after retrospectives. 

He presented how the external view made them realize how much they had actually accomplished in terms of implementing actions and it was described as``Positive to have someone from the outside as they are felt more independent''. How it had helped them realize they had a negative focus during the retrospective. He felt that already people were more conscious of being positive and there was less a ``funeral'' atmosphere.  


It was also presented how single loop and double loop had been analyzed and that they for the future would focus more on double-loop learning. 

As a final remark he described that they would make a new KIP dashboard for displaying positive actions, double loop and closed actions. To help keep a focus for the retrospectives and improvement opportunities.

The presentation reflected most of the results from the internal retrospective held and a summary of the report can be seen in \autoref{table:Retrospective-Feedback-four}. 

\begin{table}[!h]
	\begin{center}
	\caption{Retrospective feedback report summary}
	\label{table:Retrospective-Feedback-four}
	\begin{tabular}{ l  p{0.5	\textwidth} }
	\hline
	Issue & Description \\
	\hline
	Double loop & Aim to continue finding root causes instead of fixing symptoms \\
	Track more key data & The team decided to add fields in the report describing the nature and depth of the action decided on. Where nature could be positive or negative, and depth could be single or double loop.\\
	Numerator to record positives & Aiming to improve the tracking of positives the team decided to add a numerator to record positives \\
	Dashboard creation & The team decided to create and use a dashboard with information similar to the information presented in this thesis.  \\
	Completed actions & In order to track what actions are completed they decided to keep them as open in their systems, as well as add a new field to these actions for ``improvement management'' \\
	\hline
	\end{tabular}
	\end{center}
\end{table}

\subsection{Learning Obstacles}
\label{question-14}
According to the breadth study participants there were several obstacles that could prohibit learning. 

Low enthusiasm for the retrospective within the team was seen as an obstacle by several teams. The reasoning was that the persons with the low enthusiasm would rather do something else. 

Another obstacle was described as change. The retrospectives had to return some value in terms of procedural change. If nothing got done it would create low enthusiasm as there was no value to the retrospective. This could create a negative feedback loop as described above. Obstacles that supported this was mentioned by other teams. A long backlog for a sprint made it harder to take decisions, as a lot of energy was required to accomplish the actions that would be put there. 

Another obstacle to learning were the external parties. The project leader in team Alfa told us that some decisions and actions were agreed upon by the team, but customers, leadership or other parties sometimes could hinder the team from following through the action. 

Team Golf, who was a distributed team, added large physical distances as an obstacle. It made communication harder and thus could be a hinder to learning as information and feedback could be lost.

\begin{table}[!h]
	\begin{center}
	\caption{Obstacles for learning}
	\label{table:learning-obstacles}
	\makebox[\textwidth]{%
		\begin{tabular}{ l  p{0.5\textwidth}  p{0.23\textwidth}}
			Obstacle & Description & Team \\
			\hline
			Low enthusiasm & Participants finds the retrospective a waste of time and would rather do something else & Charlie \\
			Change & Without process improvement and visible changes the enthusiasm for the retrospective would decrease between the participants. & Echo, Alfa, Charlie, Foxtrot \\
			External parties &  Leadership, customers or other third parties could reject the teams from implementing actions & Alfa \\
			Large physical distance & The team is distributed over large distances, making communication complicated & Golf \\
		\end{tabular}
	}
	\end{center}
\end{table}

\subsection{Learning Enhancing Factors and Potential Improvements}
\label{question-17}
When asked about what attributes, that existed within the teams, contributed to learning the feeling of ownership to the process was described as a very strong factor by multiple interviewees. It was even described as critical to the success of a retrospective by the SCRUM master of team Delta. 

The developer from team Echo expressed the high impact a SCRUM master had on the team's appreciation of the retrospective by ensuring that retrospective tasks were completed. The attitude of wanting to improve was also described as very important.

\label{question-18}
On potential improvements that would increase the potential for learning one interviewee said he would like to improve process of ensuring tasks set during a retrospective were completed. 

Also expressed was a desire to be able to visualize issues during the retrospective. Team Charlie's SCRUM master intended to buy a white board that could be placed visibly to the team with a list of intended retrospective improvements. Team Zulu would do this as well, through implementation of a dashboard. 

The project leader of team Alfa only held retrospective when the need arose, and thought that more frequent retrospectives might be beneficial. As previously mentioned team Alfa held retrospectives both after releases and when requested by one of the team members. The project leader was satisfied with the feature driven retrospective, but told us that he thought having request retrospectives more frequent could be beneficial as these retrospectives were more thorough. 

Another potential improvement discussed was the inclusion of external parties, in order gain an outsider perspective of why they did the things they did. 

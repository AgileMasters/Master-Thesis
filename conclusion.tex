\chapter{Conclusion and Future Work}
\section{Conclusion}
Throughout this thesis we have conducted an empirical study, through a multiple case-study of mature agile development teams, investigating the outcome returned from the retrospective in terms of organizational learning and retrospective characteristics, and thereby answering Dingsøyr and Dybå’s call \cite{Dyba2008} for empirical studies into mature agile development teams.

The studies revealed that todays retrospective practices is approximating an organizational learning II system as described by Argyris and Schön \cite{Argyris1996}. Most of the governing values are already introduced in todays practice. We have seen that single-loop learning is the primary learning occurring, even though double-loop learning is the expected outcome from organizational learning II systems. We have identified one barrier to double-loop learning which consists of several factors. Through our depth study we have seen that reflection on ones own learning helps give focus to the retrospective and lower this barrier and we have proposed a method to help achieve this. 

For the current characteristics of todays retrospective we have seen that the outcome of the practice is improvement opportunities and learning which could result in improved efficiency, increased enthusiasm and adaptation of work-processes, work environment, and product quality. For the processes of the practice we have identified a feedback loop with a barrier for implementation of improvement opportunities, depending on team commitment for implementation and enthusiasm for the retrospective. 

Finally todays retrospective practices provide agile development teams to adapt their current work-practices and enables them to learn from past development iterations and thus provide the means for identifying improvements opportunities and improve from them. 

\section{Future Work}
For future work we would recommend conducting similar studies in other parts of the world to increase sample size and further verify the results of this thesis. Research investigating the method of meta-retrospectives and its value would also help verify this research. Other future work would be to determine the value of double-loop learning to help identify which issues should be identified down to their underlying influences. Finally we did some work that examined the relationship between shared mental models and the agile retrospective, but due to time constraints we were not able to perform this examination to what we think it is it's full potential. Thus this remains an area for potential future work \todo[inline]{Teoretisk retning, hva tror dere at kan læres av å bruke SMM}.

\todo[inline]{Hva er de viktigte kriteriene for [ velge case om noen vil jobbe videre]}
\todo[inline]{Forslag: -Nye studier: samme tema, -Studie på forslått praksis, + Ny teoretisk retning,}

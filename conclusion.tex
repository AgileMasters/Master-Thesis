\chapter{Conclusion and Future Work}
\section{Conclusion}
Throughout this thesis we have conducted an empirical study, of mature agile development teams. Investigating the outcome returned from the retrospective in terms of organizational learning and retrospective characteristics. Thereby answering Dingsøyr and Dybå’s call \cite{Dyba2008} for empirical studies into mature agile development teams. For the practitioners we have investigated the outcome of the characteristics and proposed a set of guidelines which could help improve their practice.

For the current characteristics of todays retrospective we have seen that the outcome of the practice is improvement opportunities and learning which could result in improved efficiency, increased enthusiasm and adaptation of work-processes, work environment, and product quality. For the processes of the practice we have identified a feedback loop with a barrier for implementation of improvement opportunities, depending on team commitment for implementation and enthusiasm for the retrospective. 

\begin{itemize}
\item Retrospectives allow teams to improve their current work-practices through learning, team commitment and investigation of past development phases.
\end{itemize}

The studies revealed that todays retrospective practices is a learning practice where teams are able to test their current work-practices, learn from them and improve from them. Which means that agile development teams that practice retrospectives is approximating an organizational learning II system as described by Argyris and Schön \cite{Argyris1996}. Where most of the governing values are already introduced to current practice. We have seen that single-loop learning is the primary learning occurring, even though double-loop learning is the expected outcome from organizational learning II systems. We have identified one barrier to double-loop learning which consists of several factors. Through our depth study we have seen that reflection on ones own learning helps give focus to the retrospective and lower this barrier and we have proposed a method to help achieve this. 

\begin{itemize}
\item Learning in the present retrospective practices are primary single-loop learning, however the learning environment could facilitate double-loop learning.
\end{itemize}

Previous critique \cite{Drury2012} have stated that the retrospective does not provide any changes to the work environment and we have observed that this could be the case if the team are not able to overcome the team commitment barrier for implementation of improvement opportunities. Further we would say that the main problem for current retrospective practice, along with the team commitment barrier, is the development team's inability to reflect on their own learning and learning processes. 

Finally todays retrospective practices provide agile development teams the ability to adapt their current work-practices and enables them to learn from past development iterations and thus provide the means for identifying improvement opportunities and improve from them. 

\section{Future Work}
For future work we would recommend conducting similar studies in other parts of the world to increase sample size and further verify the results of this thesis. Research investigating the method of meta-retrospectives and its value would also help verify this research and hopefully provide some value for practitioners. Other future work would be to determine the value of double-loop learning to help identify which issues should be identified down to their underlying influences. 

If someone chooses to perform similar studies they should have the following criteria in mind. First of all it should be an empirical study\cite{Dyba2008}. If a breadth study is to be conducted we would recommend having many teams from different types of projects and other parts of the world than Scandinavia. Teams participating in large scale agile projects, teams that are distributed across both small and large distances, teams who do continuous delivery, teams who do cyclic delivery, teams with consistent availability of members and the opposite are some examples of teams and projects. In general to get a wide understanding of how different types of teams are practicing retrospectives. For a depth study we would recommend working with teams that are open in both communication and information sharing. As this will form the cornerstone of the study. Further both the team and the researchers should both agree to the analysis method used. If a similar analysis like our content analysis should be conducted we would recommend the researchers to add those categories we have missed, like flow, work-environment and hotfix which in hindsight were missed by the authors. 

Finally we did some work that examined the relationship between shared mental models and the agile retrospective, but due to time constraints we were not able to perform this examination to what we think is its full potential. However what we did manage to investigate some of it and this is described in appendix \label{app:smm}. This remains an area for potential future work. Such a study could potentially lead to a deeper understanding of how tacit knowledge exists within a retrospective team and could possibly be used to improve the retrospective practice even further. 

\chapter{Introduction}
\todo{Write Introduction}
\section{Background}
\todo[inline]{Write background}
\section{Research Objective}
\todo[inline]{Write Research Objective}
\section{Agile Development}
\todo[inline]{Theory on Agile Development}
\section{Retrospective}
\todo[inline]{Theory and some common techniques for retrospectives, refer to self preproject}
\section{Organizational Learning}
Organizational learning is in simple terms how an organization is able to acquire, store and utilize knowledge existing within an organization. In a more academic setting we can use Argyris and Schön definition \cite{Argyris1996} for organizational learning: 

\begin{quote}
	Organizational learning occurs when individuals within an organization experience a problematic situation and inquire into it on the organization's behalf. They experience a surprising mismatch between expected and actual results of action and respond to that mismatch through a process of thought and further action that leads them to modify their images of organization or their understandings of organizational phenomena and to restructure their activities so as to bring outcomes and expectation into line, thereby changing organizational theory-in-use. In order to become organizational, the learning that results from organizational inquiry must become embedded in the images of organization held in its members' minds and/or in the epistemological artifacts (the maps, memories, and programs) embedded in the organizational environment. 
\end{quote}

Organizational learning can be applied to groups of people over different sizes. One can use it to analyze huge organizations consisting of many actors in different roles, or small groups of people working close together. For this research we are going to apply the organizational learning at an agile development team. 

Different types of frameworks exist for organizational learning. Levitt and March uses a framework that see learning in organizations as encoding inferences from history into routines that guide behavior \cite{Levitt1988}. The retrospective is a process of shared learning for an agile development team learning from the last iteration of development. Investigating learning using this framework won't yield much as the retrospective already is seen as a collective learning activity \cite{Dingsoyr2004}. 

Argyris and Schön \cite{Argyris1996} uses two different models to determine how an organization learns through individuals, Model I and Model II. As agile teams often are limited to a close group of individuals we will apply Argyris and Schön's models. To understand these models we first need to investigate two types of learning: Single-loop learning and Double-loop learning. Further in this section we will explain these types of learning as well as Model I and Model II. We will also look at Triple-loop learning as it is the concept of learning about learning. 

\subsection{Single-loop}

\subsection{Double-loop} % (fold)
\label{sub:double_loop}

% subsection triple_loop (end)
\subsection{Model I} % (fold)
\label{sub:model_i}

% subsection model_i (end)
\subsection{Model II} % (fold)
\label{sub:organizational_learning_ii}

% subsection organizational_learning_ii (end)
% subsection double_loop (end)
\subsection{Triple-loop} % (fold)
\label{sub:triple_loop}

\section{Shared Mental Models}
\todo[inline]{Theory on Shared Mental Models}
\clearpage


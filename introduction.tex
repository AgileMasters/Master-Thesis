\chapter{Introduction}
\todo{Write Introduction}
\section{Background}
\todo[inline]{Write background}
\section{Research Objective}
\todo[inline]{Write Research Objective}
\section{Agile Development}
\todo[inline]{Theory on Agile Development}
\section{Retrospective}
\todo[inline]{Theory and some common techniques for retrospectives, refer to self preproject}
\section{Organizational Learning}
Organizational learning is in simple terms how an organization is able to acquire, store and utilize knowledge existing within an organization. In a more academic setting we can use Argyris and Schön definition \cite{Argyris1996} for organizational learning: 

\begin{quote}
	Organizational learning occurs when individuals within an organization experience a problematic situation and inquire into it on the organization's behalf. They experience a surprising mismatch between expected and actual results of action and respond to that mismatch through a process of thought and further action that leads them to modify their images of organization or their understandings of organizational phenomena and to restructure their activities so as to bring outcomes and expectation into line, thereby changing organizational theory-in-use. In order to become organizational, the learning that results from organizational inquiry must become embedded in the images of organization held in its members' minds and/or in the epistemological artifacts (the maps, memories, and programs) embedded in the organizational environment. 
\end{quote}

\subsection{Single-loop}
\subsection{Double-loop} % (fold)
\label{sub:double_loop}

% subsection double_loop (end)
\subsection{Triple-loop} % (fold)
\label{sub:triple_loop}

% subsection triple_loop (end)
\subsection{Organizational Learning II} % (fold)
\label{sub:organizational_learning_ii}

% subsection organizational_learning_ii (end)

\section{Shared Mental Models}
\todo[inline]{Theory on Shared Mental Models}
\clearpage

